\documentclass{article}
\usepackage{amssymb, amsfonts, amsmath, amsthm}
\usepackage{hyperref}
\usepackage{enumerate}
%\usepackage{graphicx}
%\usepackage{url, multicol}

\usepackage[active]{srcltx}

\def\thetitle{An upper bound for curvature integral}
\def\theauthors{Anton Petrunin}

\hypersetup{
pdftitle={\thetitle}
pdfauthor={\theauthors}
}


\def\noi{\noindent}
\def\CC{\mathbb{C}}
\def\PP{\mathbb{P}}
\def\NN{\mathbb{N}}
\def\RR{\mathbb{R}}
\def\ZZ{\mathbb{Z}}
\def\QQ{\mathbb{Q}}

\def\cA{{\mathcal A}}
\def\cB{{\mathcal B}}

\def\eps{\varepsilon}
\def\ge{\geqslant}
\def\le{\leqslant}
\def\phi{\varphi}
\def\i{\subset}
\def\l{\left}
\def\r{\right}
\def\<{\langle}
\def\>{\rangle}

%operators
\def\Const{\operatorname{Const}}
\def\area{\operatorname{area}}
\def\vol{\operatorname{vol}}
\def\diam{\operatorname{diam}}
\def\codim{\operatorname{codim}}
\def\dim{\operatorname{dim}}
\def\dir{\operatorname{dir}}
\def\dist{\operatorname{dist}}
\def\GHto{\buildrel \mathrm{GH} \over \longrightarrow}
\def\pack{\operatorname{pack}}
\def\Ric{\operatorname{Ric}}
\def\Rm{\operatorname{Rm}}
\def\Sc{\operatorname{Sc}}

\mathsurround=1pt
\newtheorem{thm}{Theorem}[section]
\newtheorem{cor}[thm]{Corollary}
\newtheorem{quest}[thm]{Question}
\newtheorem{lem}[thm]{Lemma}
\newtheorem{prop}[thm]{Proposition}
\newtheorem{defn}[thm]{Definition}
\newtheorem{conj}[thm]{Conjecture}
\newtheorem{prob}[thm]{Problem}
\newtheorem{property}[thm]{Property}
\newtheorem{Example}[thm]{Example}
\newenvironment{ex}{\begin{Example}\rm}{\end{Example}}
\newtheorem{Counterexample}[thm]{Counterexample}
\newenvironment{coex}{\begin{Counterexample}\rm}{\end{Counterexample}}
\newtheorem{remark}[thm]{Remark}
\newenvironment{rmk}{\begin{remark}\rm}{\end{remark}}
\newtheorem{Fact}[thm]{$\!\!\!$}
\newenvironment{fact}{\begin{Fact}\rm}{\end{Fact}}
\newtheorem{Nothing}[thm]{$\!\!\!$}
\newenvironment{nothing}{\begin{Nothing}\rm}{\end{Nothing}}


\begin{document}

\title{\thetitle}
\author{\theauthors}
\date{}
\maketitle

\begin{abstract}
Here I show that the integral of scalar curvature of a closed Riemannian manifold
can be bounded from above in terms of its dimension, diameter,
and a lower bound for sectional curvature.
\end{abstract}

\section{Introduction.}

Let us start with a formulation of the main result:

\begin{Nothing}{\bf Theorem.}\label{main}
Let $M$ be a complete Riemannian $m$-manifold with sectional curvature $\ge -1$.
Then for any $p\in M$
$$\int_{B_1(p)}\Sc\le \Const(m),$$
where $B_1(p)$ denotes the unit ball centered at $p\in M$ and $\Sc$ --- scalar curvature of $M$.
\end{Nothing}

This result is optimal, as one can see in the following example:
Consider a convex polyhedron $P$ in $\RR^{m+1}$, and let $\partial P_\eps$ denote the surface of its $\eps$-neighborhood (if necessary, one can smooth $\partial P_\eps$ to make it $C^\infty$-smooth).
Then $\partial P_\eps$ are Riemannian $m$-manifolds with non-negative sectional curvature and the integral
$\int_{\partial P_\eps}\Sc$
remains nearly constant for small $\eps$, while for any $\theta>0$, the integral $$\int_{\partial P_\eps}|\Sc|^{1+\theta}\to \infty\ \ \text{as}\ \  \eps\to 0.$$

Note that if one takes $P$ of codimension 2 or 3, this construction gives an example of families of collapsing Riemannian manifolds $\partial P_\eps$ with integral $\int_{\partial P_\eps}\Sc$ bounded below by a positive constant.
The Corollary \ref{cor} below shows that this is no longer possible if the collaps drops dimension by $3$ or more.

\begin{nothing}{\bf Variations and generalizations.}
The Bishop--Gromov inequality implies that any ball of radius $R$ in a complete Riemannian $m$-manifold with sectional curvature $\ge -1$ can be covered by $\exp(4m R)$ unit balls, so subsequently this theorem can be generalized the following way:

$$\int_{B_R(p)}\Sc\le \Const(m)\exp(4m R).$$
On the other hand, the rescaling shows that
for radius $R<1$,
$$\int_{B_R(p)}\Sc\le \Const(m)R^{m-2}.$$
All these together lead to
\end{nothing}
\begin{Nothing}{\bf Corollary} \label{cor} Let $M$ be a complete Riemannian
$m$-manifold with sectional curvature $\ge -1$ and $p\in M$.
Then $$\int_{B_R(p)}\Sc\le \Const(m)R^{m-2}\exp(4m R).$$
\end{Nothing}

Also, it is easy to see that if a Riemann curvature tensor $\Rm_p$ at point $p$ of a Riemannian $m$-manifold has all sectional curvatures
$\ge 0$, then
$$|\Rm_p|\le \Sc_p.$$
It follows that if instead we have all sectional curvatures
$\ge -1$, then
$$|\Rm_p|\le \Sc+m^2.$$
Therefore, the last corollary also implies that
$$\int_{B_R(p)}|\Rm|\le \Const'(m)R^{m-2}\exp(4m R),$$
where
$$\Const'(m)=\Const(m)+m^2.$$

\begin{nothing} I believe that in 1996 Gromov asked me whether the integral
of the scalar curvature could be bounded in terms of a diameter and lower curvature bound.
However, after 10 years I can not be sure that it was exactly the question he asked.

My proof is similar to the Perelman's proof of continuity of the integral of
the scalar curvature
$$\mathcal F(M)=\int_M \Sc$$
on the set of all Riemannian $m$-manifolds with a uniform lower curvature bound.
Perelman's proof uses exhausting of the manifold by convex hypersurfaces (this proof is included as an appendix in \cite{Pet}).
In fact just a slight modification of Perelman's proof makes it possible to prove
the main theorem in the non-collapsing case, i.e. in the case one adds a lower bound
for the volume of the ball, $$\vol(B_1(p))\ge v_0>0$$
as an extra condition.

In addition I use a special form of Bochner's formula (see \ref{boch});
a similar idea was used by Sergei Buyalo for lower estimates of
the integral of the scalar curvature on a 3-manifold \cite{Buy}.
This formula makes it possible to estimate curvature not only in tangent
but also in normal directions to the exhausting surfaces
(which is necessarily to make the proof work in the collapsing case).
Also, I had to use exhaustions by nested sequences of semiconvex surfaces,
which causes additional technical difficulty.

The rest of the proof is a combination of converging-rescaling technique.
\end{nothing}
\begin{nothing}{\bf Frame of the proof.}
Arguing by contradiction, assume that there is a sequence of $m$-manifolds $(M_n,p_n)$ with $K\ge -1$ such that $$\int_{B_1(p_n)}\Sc\to\infty.$$
Applying Gromov's compactness theorem we can assume that $(M_n,p_n)$ converges to an Alexandrov space $(A,p)$.
Using the induction hypothesis (the fact that a slightly more general statement
is true for all smaller dimension) we prove that the scalar curvature on $M_n$
weakly converges to a measure on $A$ which is finite everywhere except
a finite number of points.
We choose one of these points $s\in A$ and blow up $M_n$ with carefully chosen
marked points $s_n\in M_n$ near $s$ to make curvature distribution visible,
then pass to the new limit space $(A',p')$.
Repeat for $(A',p')$ the same procedure as for $(A,p)$ and so on.
The statement follows since there will be only a finite number of such repetitions.
The last statement follows from the fact that $(A',p')$ is in a certain sense,
by certain amount bigger than $(A,p)$ and it can not be bigger than $(\RR^m,0)$.
\end{nothing}

\begin{nothing}{\it Acknowledgements.}
I would like to thank Nina Lebedeva and Yuri Burago for their comments to the original version of
this article and noticing mistakes;
I am also thankful to Fyodor Zarkhin for correcting my English.
\end{nothing}

\section{Bochner formula.}

This section provides a form of Bochner formula which links the integral of
the scalar curvature of a family of hypersurfaces with the integral of the Ricci
curvature in their normal directions.

\begin{nothing}{\bf Notation.}
Let $M^{m+1}$ be a Riemannian $(m+1)$-manifold.
Assume $f:M^{m+1}\to \RR$ is a smooth function without critical values on $[a,b]\i \RR$ and its level sets $L_t=f^{-1}(t)$ are compact for any $t\in [a,b]$.
Let
\begin{enumerate}[(i)]
\item $u=\nabla f/|\nabla f|$ be the unit vector field normal to $L_t$,
\item $\{e_0,e_1,e_2,...,e_m\}$ an orthonormal frame such that $e_0=u$ and each
$e_i$, $i>0$ is chosen in a principle directions of
$L_{f(x)}$ at $x$ such that the corresponding principle curvatures
$$\kappa_i=\<\nabla_{e_i}u,e_i\>.$$
form a nondecreasing sequence
$$\kappa_1(x)\le\kappa_2(x)\le...\le\kappa_m(x).$$
\item $H(x)=\kappa_1(x)+\kappa_2(x)+...+\kappa_m(x)$ mean curvature of $L_{f(x)}$ at
$x\in L_{f(x)}$,
\item $G(x)=2\sum_{i<j}\kappa_i(x)\kappa_j(x)$ the external term in the Gauss formula
for the scalar curvature of $L_t$, i.e.
$$\Sc_{L}=2\sum_{i<j}\<R_M(e_i,e_j)e_j,e_i\>+G=\Sc_M-2\Ric_M(u,u)+G.$$
\end{enumerate}
\end{nothing}

\begin{Nothing}{\bf Bochner formula.}\label{boch} Using this notation one can write the Bochner formula in the following form:
$$\int_{f^{-1}([a,b])} \Ric_M( u, u)=\int_{f^{-1}([a,b])} G+\int_{L_a}H-\int_{L_b}H \qquad(*) $$
\end{Nothing}

\

\paragraph*{Proof.} Let us write relative Bochner formula for the vector field
$u=\nabla f/|\nabla f|$ in the domain $f^{-1}([a,b])$:
$$
\int_{f^{-1}([a,b])}\langle Du,Du \rangle
-
\langle \nabla u,\nabla u \rangle
=
\int_{f^{-1}([a,b])} \Ric_M( u, u)-\int_{L_a}H+\int_{L_b}H.
$$
Since $e_0=u$ and $\<\nabla_u u,u\>=0$,
$$ Du=\sum_{i=0}^{m} e_i\cdot \nabla_{e_i}u=
\sum_{i=1}^{m}\kappa_i e_i\cdot e_i+u\cdot \nabla_{u}u=
\sum_{i=1}^{m}\kappa_i+u\wedge\nabla_{u}u,$$
here ``$\,\cdot\,$'' denotes the Clifford multiplication.
Applying again that $\<\nabla_u u,u\>=0$,
$$ \langle Du,Du \rangle=\l(\sum_{i=1}^{m}\kappa_i\r)^2+|\nabla_{u}u|^2.$$
On the other hand
$$\nabla u=\sum_{i=1}^m\kappa_ie_i\otimes e_i+\nabla_u u\otimes u,$$
hence
$$\langle\nabla u,\nabla u\rangle =
\sum_{i=1}^{m}\kappa_i^2+|\nabla_{u}u|^2.$$

Therefore

$$\langle D u,D u\rangle-\langle \nabla u,\nabla u \rangle =2\sum_{i<j}\kappa_i\kappa_j=G.$$

I.e., in this notation the Bochner formula boils down to $(*)$.\qed

\section{Constructions in Alexandrov's geometry.}

This section contains all the necessary technical results in Alexandrov's geometry,
mainly related to {\it corner surfaces}.
A corner surface is a generalization of semi-convex hypersurfaces to all codimension.
Roughly speaking it is an intersection of semi-convex hypersurfaces which have acute angles
between each other.

These level sets will be used in the proof in the same manner as convex hypersurfaces in Perelman's original proof.

Everywhere in this section we use notation and conventions as in  \cite{petrunin-2}.

The following definition is very similar to the definition of strained (bursted) points
in \cite{BGP}, but I use it for submanifolds and have to add an extra parameter
$\ell$ which describes how wide strainers should be.

\begin{Nothing}{\bf Definition.} A subset  $M$ of an Alexandrov space $N$
is a \emph{$(k,\delta,\ell)$-corner surface} if there is a collection of
$1/\ell$-concave functions $f_i, g_i:N\to \RR$, $i\in \{1,..,k\}$
(called \emph{strainers} of $M$), defined in an $\ell$-neighborhood of $M$ such that

$$M=\{x\in N|f_i(x)=0\ \text{for all} \  1\le i\le k\},$$
the collection $f_i, g_i:N\to \RR$ is $\delta$-strained, i.e.
\begin{enumerate}[(i)]
\item all $f_i$, $g_i$ are 1-Lipschitz,
\item for all $i\not=j$,
$$|df_i(\nabla g_j)|,\,|dg_i(\nabla f_j)|,\,|dg_i(\nabla g_j)|,\, |df_i(\nabla f_j)|\le\delta$$
%$$\angle(\nabla f_i,\nabla f_j),\angle(\nabla f_i,\nabla g_j), \angle(\nabla g_i,\nabla g_j)\in[\pi/2- \delta;\pi/2+ \delta],$$
\item $df_i(\nabla g_i), dg_i(\nabla f_i)\le -1+2\delta$
%$\angle(\nabla f_i,\nabla g_i)\ge \pi-\delta$.

\noi \!\!\!\!\!\!\!\!\!\!\!\!\!\!and the collection $f_i:N\to \RR$ is tight, i.e

\item for any $x\in M$ and all $i\not=j$,
$$df_i(\nabla f_j)\le 0 $$
%$$\angle(\nabla_x f_i,\nabla_x f_j)\ge\pi/2.$$
\end{enumerate}
If in addition $N$ is a smooth Riemannian manifold and all functions
$\{f_i\}$ are smooth then $M$ is called {\it smooth $(k,\delta,\ell)$-corner surface}.
\end{Nothing}

\noi{\bf Remarks.} In this definition one can take $k=0$, in this case $M=N$.

Conditions (i)--(iv) guarantee that functions $f_i$ do not have critical points and
their level sets intersect with acute angles which are close  to $\pi/2$.

Note that if one rescales the metric on $N$ with factor $\lambda$ then a
$(k,\delta,\ell)$-corner surface $M$ becomes a $(k,\delta,\lambda\ell)$-corner
surface in $\lambda N$ with strainers $\{\lambda f_i,\lambda g_i\}$.



\begin{nothing}{\bf Limits of strained level sets.}\label{conv} Assume we have a sequence of $q$-manifolds $(N_n,p_n)$ with marked points and sectional curvature $\ge -1$.
Let $M_n\i N_n$ be a sequence of $(k,\delta,\ell)$-corner surfaces  defined by collections of strainers
$$\{f_{i,n},\ g_{i,n}\},\ \  i\in\{1,\dots,k\}, \ \ f_{i,n},\,g_{i,n}:N_n\to \RR.$$
One can pass to a subsequence of $N_n$ to have convergences
\begin{enumerate}[(i)]
\item $(N_n,p_n)\GHto (N,p)$,
\item $M_n\to M\i N$,
\item  for each $i\in \{1,\dots,k\}$, $f_{i,n}\to f_i:N\to \RR$ and $g_{i,n}\to g_i:N\to \RR$.
\end{enumerate}
Clearly $N$ is an Alexandrov space of the dimension at most $q$ with curvature $\ge -1$ and
$M$ is a $(k,\delta,\ell)$-corner surface with strainers $f_i,g_i:N\to \RR$.

Moreover, if $M$ is compact and $m=q-k$, then
$$\vol_m M=\lim_{n\to\infty}\vol_m M_n.$$
This statement follows from the proof of \cite[Th. 10.8]{BGP}.
\end{nothing}

\begin{nothing}{\bf A dimension-like invariant.}\label{dir}
Here we introduce an invariant of Alexandrov spaces,
which has properties similar to the dimension, but a bit more sensitive.

Let $A$ be an Alexandrov space with dimension $\le q$.
Given a positive number $\theta>0$ let us define $\dir_\theta A$ to be
$$\dir_\theta A=\min_{x\in A}\pack_\theta \Sigma_x A$$
where $\Sigma_x A$ is the space of directions at $x\in A$ and $\pack_\theta \Sigma_x A$
denotes the maximal number of points in $\Sigma_x A$ at the distance
$> \theta$ from each other.


Clearly $\dir_\theta A$ is an integer and
$$\dir_\theta A\le \Const(\theta,\dim A)\le \pack_\theta S^q.$$
\end{nothing}



\begin{nothing}{\bf A special way to lift points.}\label{approx}
Here we introduce a special way to lift a point from an Alexandrov space to a nearby
Riemannian manifold.
More precisely, we describe a way to lift a spire (see below)
on a $(k,\delta,\ell)$-corner surface in an Alexandrov space to
a nearby $(k,\delta,\ell)$-corner surface in a nearby Riemannian manifold.

This technique will be used just once, at the very end of the proof
of the implication
B$_{m} \Rightarrow$A$'_m$ (\ref{BtoA}) of the monster-lemma \ref{monster};
for the rest, any kind of lifting will do the job.

\

Assume that $N_n\to N$ is a sequence of Riemannian $q$-manifolds with curvature $\ge -1$ converging to an Alexandrov space $N$.
Let $\delta>0$ be sufficiently small.
Let $M_n\i N_n$ be a sequence of  $(k,\delta,\ell)$-corner surfaces which converges to $M\i N$ (see \ref{conv}).

Consider a positive function $b$ on $M$, which takes maximal possible value $b(x)$ such that
$$\bigl|\nabla_y\dist_x\bigr|> 1-\delta\ \text{for any}\  y\in B_{2b(x)}(x)\backslash \{x\}\ \text{and}\ 2b(x)\le\max\{1,\delta\ell\}$$

\begin{Nothing}{\bf Definition.}\label{spire}
A point $x$ on a $(k,\delta,\ell)$-corner surface is called a spire if for any $y\in M$, $y\not=x$ we have $b(y)\le |x y|$.
\end{Nothing}

Let us show that if $x$ is a spire then there is a sequence of points $x_n\in M_n$ which converges to $x\in M$ and satisfies the following property:



\begin{Nothing}
{\bf Property.}\label{property} Let $a_n$ be the minimal number such that
$$|\nabla_y\dist_{x_n}|> 1-\delta\ \text{for all points}\ y\in N_n \ \text{such that}\  a_n<|x_n y|\le b$$
Then $a_n=0$ for arbitrarily large $n$ or if
$(N', x')$ is a partial limit of $(\frac1{a_n}N_n, x_n)$ then
$$\dir_\delta N'>\dir_\delta N \ \ \text{(see \ref{dir})}.$$

\end{Nothing}

\begin{nothing}{\bf The construction.}\label{constr}
Let us define $a(\tilde x_n)=a_{\delta,b}(\tilde x_n)$ for any $\tilde x_n\in N_n$ to be the minimal number such that
$$|\nabla_y\dist_{\tilde x_n}|> 1-\delta\ \  \text{if}\ \ a(\tilde x_n)<|\tilde  x_n y|\le b.$$
Note that for any sequences $M_n\ni \bar x_n\to x$ and $M_n\ni y_n\to y\not= x$, we have
$$\liminf_{n\to\infty}|\nabla_{y_n}\dist_{\bar x_n}|\ge |\nabla_y\dist_{x}|,$$
therefore $a(\bar x_n)\to 0$ as $n\to \infty$.

Now fix one such sequence $\bar x_n\to x$ and let
$x_n\in M_n\cap B_r(\bar x_n)\i N_n$
be a point with minimal possible $a(x_n)$,
here $r$ is a sufficiently small fixed number.

Set $a_n=a(x_n)$, since $x$ is a spire, we have $x_n\to x$ and in particular $(N_n,x_n)\to (N,x)$.

All that remains to prove is that the chosen sequence satisfies property \ref{property}.

Assume $a_n>0$ for all large $n$.
Let us pass to a subsequence so that $(\frac1{a_n}N_n,x_n)\to (N', x')$.
Clearly, for any $\theta>0$
$$\dir_\theta N'\ge \pack_\theta \Sigma_x N\ge \dir_\theta N.$$
Therefore, we only have to show  that if $\theta=\delta$ then the first inequality is strict.

If the equality takes place, then there is a point $p\in  N'$ such that $\pack_\delta \Sigma_{p}N'=\pack_\delta \Sigma_x N=s$.
Let us show that if $p_n\in \frac1{a_n}N_n$ is a sequence of points converging to
$p$ then $\frac1{a_n}a(p_n)\to0$. In particular, for large $n$ we have $a(p_n)<a(x_n)$,
which contradicts the choice of $x_n$ (here we denote by $p_n$ point in $\frac1{a_n}N_n$
as well as the respective point in $N_n$).

Choose points $q_1,q_2,..,q_s\in N$ such that $\angle q_i x q_j>\delta$ if $i\not=j$.
Let $q_{i,n}\in N_n$ be a sequence converging to $q_i\in N$.
Let $y_n\in N$ be a point such that
$|p_n y_n|_N\to 0$ and $|\nabla_{y_n}\dist_{p_n}|\le 1-\delta$.
Then it is easy to see that $\angle y_n p_n q_n\ge 2\delta$ for large $n$,
and passing to the limit $n\to\infty$ we get $\pack_\delta \Sigma_x N>s$, a contradiction.
\end{nothing}
\end{nothing}


\section{Proof of the Theorem}

\begin{nothing}{\bf Notation.}
 Let $X$ be Riemannian manifold, $x\in X$ and $\sigma$ a sectional direction at $x$.
We denote by $K_X(\sigma)$ the sectional curvature of $X$ in the direction $\sigma$.
Let us set
$$K^\pm_X(x)=\max\{0,\max_\sigma\{\pm K(\sigma)\}\},$$
where $\sigma$ runs over all sectional directions at $x$.

%\item $\Const(x,y,z)$ will denote a positive constant which depends only on $x,y,z$, it can denote different constants, even in the same formula. For example, it is OK to write
%$$\Const(x,y,z)=\Const(x,y,z)+1.$$

%\item Calligraphic letters like $\cA(x,y,z), \cB(x,y,z)$ will denote a positive constant which depends only on $x,y,z$.

\end{nothing}
%\pagebreak

Theorem \ref{main} follows from the following statement for $k=0$.

\begin{Nothing}{\bf Theorem.} There is $\delta=\delta(q,k)>0$ such that if $N$ is
a Riemannian $q$-manifold with sectional curvature $\ge -1$ and $M\i N$ is
a complete smooth $(k,\delta,\ell)$-corner surface, then
$$\int_{M\cap B_1(x)}\Sc^+_M\le \Const(q,k,\ell)\l[1+\int_{M\cap B_2(x)}K^-_M\r]$$
for any $x\in N$.
\end{Nothing}

If $\dim M\ge 3$, this follows from statement A$'_k$ of the monster-lemma \ref{monster} below and the case $\dim M= 2$ follows from 3-dimensional case for $M\times S^1\i N\times S^1$.

\begin{Nothing}{\bf Monster-lemma.}\label{monster}
There are constants $\cA_k=\cA(q,k,\ell)$, $\cA'_k=\cA'(q,k,\ell)$,  $\cB_k=\cB(q,k,\ell)$ and a sequence of small positive  constants $\delta_k$, $k=\{0,1,..,q-2\}$ such that if
\begin{description}
\item $N$ is a Riemannian $q$-manifold with sectional curvature $\ge -1$,
%\item $\delta_k=\exp(-10^{q-k}q)$ and
\item $M\i N$ be a complete  smooth  $(k,\delta_k,\ell)$-corner surface
\end{description}
then for any $k\in \{0,1,..,q-3\}$ the following statements are true:

\begin{description}

\item{A$_k$.} If $\diam M\le 1$, then
$$\int_{M}\Sc^+_M\le \cA_k\l[1+\int_{M}K^-_M\r].$$


\item{A$'_k$.} For any $x\in N$
$$\int_{M\cap B_1(x)}\Sc^+_M\le \cA'_k\l[1+\int_{M\cap B_2(x)}K^-_M\r].$$


\item{B$_k$.}
Assume that for some $x\in M$ we have $|\nabla_y\dist_x|>1-\delta_k$ for all
$y\in N$ such that $a<|x y|<2b<\max\{1,\delta\ell\}$, then
$$\int_{\dist_x^{-1}([2a,b])\cap M}\Sc^+_{M}
\le  \cB_k\l[1+\int_{\dist_x^{-1}([a,2b])\cap M}K^-_{M}\r].$$
\end{description}
\end{Nothing}




\paragraph*{Proof.} Clearly ${\rm A}'_k\Rightarrow {\rm A}_k$.
Therefore in order to prove Lemma it is enough to prove the following statements:
${\rm A}_{q-2}$ and ${\rm A}_k\Rightarrow {\rm B}_{k-1}$, ${\rm B}_k\Rightarrow {\rm A}'_{k}$ for each $k$.

\begin{nothing}{\bf $\boldmath{\text{A}_{q-2}}$.}
In this case $\dim M=2$, therefore $\Sc_M^\pm=2K_M^\pm$.
Then the statement follows from the Gauss--Bonnet formula:
$$\int_M K_M^+=\int_M K_M+ \int_M K_M^-\le 4\pi\l(1+\int_M K_M^-\r).$$
So one can take $\cA_{q-2}=8\pi$.
\qed
\end{nothing}

\begin{nothing}{\bf$\boldmath{\text{A}_{k}\Rightarrow\text{B}_{k-1}}$.} Assume $\{f_i,g_i\}$, $i=\{1,..,k-1\}$ is a set of strainers for $M$.
Let us consider the function $f:M\to \RR$ defined by the following formula:
$$f(y)=(1-\delta_k)\widetilde{\dist}_x(y)
+
\frac{\delta_k}k \sum_{i=1}^{k-1}(g_i(y)-g_i(x)).$$
where $\widetilde{\dist}_x$ denotes a smoothing of the distance function ${\dist}_x$.
On the set $\dist^{-1}([a,2b])\cap M$, we clearly have
$$1-2\delta_k\le \frac{f(y)}{\dist_x y} < 1.$$
Therefore to prove B$_{k-1}$, it is enough to show that for some constant $\cB(q,k,\ell)$ we have
$$\int_{f^{-1}([a,b])}\Sc^+_{M}
\le \cB(q,k,\ell)\l[1+\int_{f^{-1}([a,\tfrac32b])}K^-_{M}\r] \qquad (**)$$

On the set $\dist^{-1}([a,2b])\cap M$, the function $f$ behaves similarly to $\dist_x$,
in addition it is smooth and its level subsets form corner surfaces in $N$
(so we can apply A$_k$). Moreover,
$f$ satisfies the following properties (compare with \cite[11.8]{BGP})

The function $f$ \label{f-prop}is a semiconcave in a neighborhood of $f^{-1}([a,\tfrac32b])$ and for some $\alpha=\alpha(q,k,\ell)>0$ we have:
\begin{enumerate}[(i)]
\item\label{grad}$1\ge |\nabla f|>1/\alpha$ everywhere in $f^{-1}([a,\tfrac32b])$ on $M$,
in particular
$f:M\to \RR$
does not have critical values on $[a,\tfrac32b]$.
\item\label{geom} for any $t\in [a,\tfrac32b]$, the level set $L_t=f^{-1}(t)$
forms a compact smooth $(k,\delta_{k},t/\alpha)$-strained level set in $N$
and if $m=\dim L_t=q-k$, we have
\begin{enumerate}[a)]
\item $A(t)\buildrel \text{def}\over=\vol_{m} L_t\le \alpha t^{m}$.
\item $\diam L_t\le \alpha t$,
\item the principle curvatures of $L_t$ in $M$ are at most $\alpha/t$.

Namely, if $u=\nabla f/|\nabla f|\in T M$ then for any unit vector $v$ tangent to $L_t$,
$$\<\nabla_v u,v\>\le \alpha/t,$$
where $\nabla$ denotes the Levi-Civita connection on $M$.
\end{enumerate}
\end{enumerate}

To prove that $L_t=f^{-1}(t)$ is a $(k,\delta_{k},t/\alpha)$-corner surface in $N$,
it is enough to add two functions to the collection of strainers of $M$,
one can take $f_k=f$ and $g_k=\dist_{L_{(1+\eps)t}}$ for a sufficiently
small constant $\eps>0$.
To prove the volume estimate, one can argue by contradiction,
using the convergence of the volume of corner surfaces, see \ref{conv}.
Details of these proof are left to the reader.


%In the proof of $(**)$ conditions (\ref{grad}) and (\ref{geom}) will be used separately. Condition (\ref{geom}) will be used only in the end of the proof of inequality. A$_{m}$ will be used just once in the middle of proof of main inequality.

\paragraph*{Notation.}
\begin{enumerate}[(i)]
\item $\kappa_1(x)\le \kappa_2(x)\le...\le \kappa_m(x)$ are the principle curvatures
of $L_t\i M$ at $x\in L_t$ with respect to $u$.
\item $\beta:[a,b]\to \RR_+$ is an upper bound for the principle curvatures on $L_t$,
i.e. $\kappa_m(p)\le \beta_t$ for any $p\in L_t$ (for condition (\ref{geom}), one can take $\beta(t)= \alpha/t$ but to simplify calculations we will substitute it only at the very end)
\item $H(x)=\kappa_1(x)+ \kappa_2(x)+...+ \kappa_m(x)$ is the mean curvature of
$L_{t}$ at $x\in L_{t}$ and $$H^\pm(x)=\max\{0,\pm H(x)\}$$
denotes positive/negative part of $H(x)$.

Note that we choose the signs of $\kappa_i$ to have the following formula:
$$A'(t)=\int_{L_t}H/|\nabla f|.$$
\item $G(x)=2\sum_{i<j}\kappa_i\kappa_j$ is the external term in the Gauss formula
for the scalar curvature of $L_t$, i.e.
$$\Sc_{L}=2\sum_{i<j}\<R_M(e_i,e_j)e_j,e_i\>+G=\Sc_M-2\Ric_M(u,u)+G$$
for an orthonormal frame $\{e_i\}$ of the tangent space to $L$.
\end{enumerate}

\paragraph*{Trivial inequalities.}
\begin{enumerate}[(i)]
\item\label{gauss1} Let $L\i M$ be a hypersurface, by the Gauss formula we have
$$
\Sc_{L}
=
\Sc_{M}-2\Ric_{M}(u,u)+G.
$$

therefore
$$
\Sc^+_{M}
\le
\Sc^+_{L}+m(m-1)K^-_{M}+2\Ric_{M}(u,u)-G
\le$$
$$\le
\Sc^+_{L}+m^2K^-_{M}+2\Ric_{M}(u,u)-G.
$$
and
$$G\le \Sc^+_{L} + (m-1)(m-2)K^-_{M}\le$$
$$\le \Sc^+_{L} + m^2K^-_{M}$$

\item\label{gauss2} Again, by the Gauss formula we have
$$K^-_{L}\le K^-_{M}+ (H^-+m\beta)\beta.$$

\item\label{H} Clearly $H^+(x)\le m\beta_t$ for any $x\in L_t$, thus
$$\int_{L_t}H^+\le m\beta_t A(t),$$
therefore
$$\int_ {L_t}H^-\le \int_{L_t}H^-/|\nabla f|
=
\int_{L_t}H^+/|\nabla f|-\int_{L_t}H/|\nabla f|\le$$
$$\le m\alpha \beta_t A(t)-A'(t)$$
\end{enumerate}

\paragraph{\bf Intermediate inequality.} Let us prove first that
$$\int_{f^{-1}([a,b])}\Sc^+_{L}
\le \Const(q,k,\ell)\l[1+\int_{f^{-1}([a,b])}K^-_{M}\r]\qquad (\diamondsuit)$$
Indeed, since $|\nabla f|\ge 1/\alpha$,
$$\int_{f^{-1}([a,b])}\Sc^+_{L}
\le
\alpha\int_a^b d t \int_{L_t}\Sc^+_{L}
\le $$
then, applying A$_k$ for $L_t$,
$$
\le \cA(q,k,\ell)\alpha\int_a^b d t\l(1+\int_{L_t} K^-_{L_t}\r)
\le$$
then, applying the trivial inequalities above,
$$\le
\cA(q,k,\ell)\alpha\int_a^b d t\l(1+\int_{L_t} [K^-_{M}+ (H^-+m\beta_t)\beta_t]\r)
\le$$
$$
\le \cA(q,k,\ell)\alpha\l[(b-a)+\int_{f^{-1}([a,b])}K^-_{M}+m(1+\alpha)\int_a^b A(t)\beta_t^2-\int_a^b A'(t)\beta_t\r]$$
applying that $0<a<b\le 1$, $\alpha=\alpha(q,k,\ell)$, $\beta_t=\alpha/t$ and $A(t)\le \alpha t^m$ we obtain $(\diamondsuit)$.


\paragraph*{Main inequality.}
$$\int_{f^{-1}([a,b])}\Sc^+_{M}
\le $$
$$\le
\int_{f^{-1}([a,b])}\Sc^+_{L}
+m^2\int_{f^{-1}([a,b])}K^-_{M}
+2\int_{f^{-1}([a,b])}\Ric(u,u)-\int_{f^{-1}([a,b])}G
\le
$$
by the relative Bochner formula,
$$
\le
\int_{f^{-1}([a,b])}\Sc^+_{L}
+m^2\int_{f^{-1}([a,b])}K^-_{M}
+\int_{f^{-1}([a,b])}G
+2\int_{L_a}H
-2\int_{L_b}H
\le
$$
applying inequality~(\ref{gauss1}),
$$
\le
2\int_{f^{-1}([a,b])}\Sc^+_{L}
+2m^2\int_{f^{-1}([a,b])}K^-_{M}
+2\int_{L_a}H^+
+2\int_{L_b}H^-
\le
$$
then applying the intermediate inequality $(\diamondsuit)$ and estimates for
the integrals of $H^\pm$ we get
$$\le\Const(q,k,\ell)\l[1+\int_{f^{-1}([a,b])}K^-_{M}\r] +m\beta_a A(a)+m\alpha \beta_b A(b)-A'(b)\le$$
$$\le \Const(q,k,\ell)\l[1+\int_{f^{-1}([a,b])}K^-_{M}\r]-A'(b).$$
In particular, for any $\tau\in [b,\tfrac32b]$ we have
$$\int_{f^{-1}([a,b])}\Sc^+_{M}
\le \Const(q,k,\ell)\l[1+\int_{f^{-1}([a,\tfrac32b])}K^-_{M}\r]-A'(\tau).$$
Since $0\le A(t)\le \alpha t^m$ (see page~\pageref{f-prop}), we get that for some $\tau\in [b,\tfrac32b]$, $(-A'(\tau))\le 2\alpha b^{m-1}$. Therefore we obtain $(**)$.\qed



\end{nothing}
\begin{nothing}{\bf$\boldmath{\text{B}_{k} \Rightarrow\text{A}'_k}$.}\label{BtoA}
We will argue by contradiction,
assuming that A$'_k$ is false.
Then there is a sequence of $q$-manifolds $N_n$ with sectional curvature $\ge -1$
with $(k,\delta_k,\ell)$-corner surfaces $M_n\i N_n$ defined by strainers $\{f_{n,i},g_{n,i}\}$
and a sequence of points $x_n\in N_n$ such that
$$\frac{\int_{M_n\cap B_1(x_n)}\Sc^+_{M_n}}{1+\int_{M_n\cap B_2(x_n)}K^-_{M_n}}\to\infty \ \ \mbox{as}\ \ n\to\infty.\qquad(\bigstar)$$
One can pass to a subsequence of $N_n$ to have the following convergences (see \ref{conv}):

\begin{enumerate}[(i)]
\item $(N_n,x_n)\to (N,x)$, $N$ is an Alexandrov space of dimension at most $q$ and curvature $\ge -1$,
\item  for each $i\in \{1,\cdots,k\}$, $f_{i,n}\to f_i:N\to \RR$ and $g_{i,n}\to g_i:N\to \RR$.
\item $M_n\to M\i N$, where $M$ is a $(k,\delta_k,\ell)$-corner surface with strainers $f_i,g_i:N\to \RR$.
\end{enumerate}

\paragraph*{Case with no spires.} Assume that $M$ has no spires (see \ref{spire}).
Then, since $M\cap \bar B_1(x)\i N$ is compact, it can be covered by
a finite number of annuli
$$\operatorname{Ann}_{x_i}=\{y\in M|0<|x_i y|<b(x_i)\}, \ i=\{1,2,..,s\}$$
Choose a sequence $x_{i,n}\in M_n$ converging to $x_i\in M$.
Set $b_i=b(x_i)$ and $a_{i,n}=a_{\delta_k,b_i}(x_{i,n})$ (see \ref{constr}).
As it is shown in \ref{constr}, $a_{i,n}\to 0$ as $n\to\infty$.
Then, if $n$ is large, applying B$_k$ for each $x_{i,n}$ with pair $(a_{i,n},b_i)$, we get
$$\int_{M_n\cap B_1(x_n)} \Sc^+_{M_n}
\le$$
$$\le
\sum_{i=1}^s
\int_{\{y\in M_n|2a_{i,n}<|x_{i,n}y|<b_i\}} \Sc^+_{M_n}\le$$
$$\le\sum_{i=1}^s
\cB_k\l[1+\int_{\{y\in M_n|a_{i,n}<|x_{i,n}y|<2b_i\}} K^-_{M_n}\r]\le$$ $$\le \cB_k s\l[1+\int_{M_n\cap B_2(x_n)} K^-_{M_n}\r].$$
Therefore, we get a contradiction with $(\bigstar)$.

\paragraph*{Case with spires.}
First let us note that one can remove from $M\cap\bar B_1(x)$ a finite set of spires
(see \ref{approx}), such that the remaining part can be covered by finite number of annuli:
$$(M\backslash \{x_1,x_2,\dots,x_s\})\cap\bar B_1(x)\i\bigcup_{i=1}^S \operatorname{Ann}_{x_i},\ \  S\ge s.$$
So centers $x_i$ for $i\le s$ are all spires and for $i> s$ are not.
Then applying  the same estimate as before, we get
$$\int_{M_n\cap B_1(x_n)} \Sc^+_{M_n}-
\sum_{i=1}^s
\int_{\{y\in M_n:\ |\tilde x_{i,n}y|<2a_{i,n}\}} \Sc^+_{M_n}
\le$$
$$\le
\sum_{i=1}^S
\int_{\{y\in M_n:\ 2a_{i,n}<|\tilde x_{i,n}y|<b_i\}} \Sc^+_{M_n}\le $$
$$\le
\sum_{i=1}^S
\cB_k\l[1+\int_{\{y\in M_n:\ a_{i,n}<|\tilde x_{i,n}y|<b_i\}}K^-_{M_n}\r]\le $$
$$\le
\cB_k S\l[1+\int_{M_n\cap B_2(x_n)}K^-_{M_n}\r].
$$
From $(\bigstar)$ we can find $i\le s$ such that
$$\frac{\int_{M_n\cap B_{2a_{i,n}(\tilde x_{i,n})}}\Sc^+_{M_n}}
{1+\int_{M_n\cap B_{4a_{i,n}(\tilde x_{i,n})}}K^-_{M_n}}\to\infty.$$
Clearly in this case $a_{i,n}>0$ for all large $n$ and $a_{i,n}\to 0$ as $n\to\infty$.

Let us choose such $i$ and pass to shorter notations
$$x:=x_i,\ b :=b_i,\ x_n:=x_{i,n},\ a_n:=a_{i,n}\ \text{and so on}.$$

Consider rescalings $N'_n=\frac1{2a_n} N_n$.
Let us denote by $M'_n$ the image of $M_n$ in $N'_n$ and for the rest, let us leave the same symbol for corresponding objects in $N_n$ and $N'_n$ (it is an abuse of notation).
Note that
$$\int_{M'_n\cap B_1(x_n)}
\Sc^+_{M'_n}=\frac1{(2a_n)^{n-2}}\int_{M_n\cap B_{2a_{n}}(x_n)}\Sc^+_{M_{n}},$$
$$\int_{M'_n\cap B_2(x_n)}K^-_{M'_n}=\frac1{(2a_n)^{n-2}}\int_{M_n\cap B_{4a_{n}}(x_n)}K^-_{M_n}.$$
Therefore, since $a_n\to 0$,
$$\frac{\int_{M_n\cap B_{2a_n}(\tilde x_n)}\Sc^+_{M_n}}
{1+\int_{M_n\cap B_{4a_{n}}(\tilde x_n)}K^-_{M_n}}\to\infty\ \ \mbox{as}\ \ n\to\infty.$$
This implies
$$\frac{\int_{M'_n\cap B_{1}(\tilde x_{n})}\Sc^+_{M'_n}}
{1+\int_{M'_n\cap B_2(\tilde x_{n})}K^-_{M'_n}}\to\infty\ \ \mbox{as}\ \ n\to\infty.$$

Passing to a subsequence if necessarily, we can assume that $(N'_n,x'_n)\to (N',x')$ and repeat for $N'$ the same procedure as for $N$.
It only remains to show
\end{nothing}

\begin{Nothing}{\bf Claim.} There may be only a finite number of such repetitions.
In other words, after a finite number of repetitions, we will get a case with no spires.
\end{Nothing}

This claim follows from the fact that $\dir_{\delta_k} N'>\dir_{\delta_k} N$ (see property \ref{property}) plus the fact that $\dir_{\delta_k} N$ is an integer and $\dir_{\delta_k} N\le \dir_{\delta_k}(\RR^q)<\infty$, see \ref{dir}.

\begin{thebibliography}{References}

\bibitem[BGP]{BGP}Burago, Yu.; Gromov, M.; Perelman, G. \textit{A. D. Aleksandrov spaces
with curvatures bounded below.} (Russian)  Uspekhi Mat. Nauk  47  (1992),  no.
2(284), 3--51, 222;  translation in  Russian Math. Surveys  47  (1992),  no. 2,
1--58

\bibitem[Buyalo]{Buy} Buyalo, S. V. \textit{Some analytic properties of convex sets in Riemannian spaces.} (Russian)  Mat. Sb. (N.S.)  107(149)  (1978), no. 1, 37--55, 159.

\bibitem[P-2003]{Pet} Petrunin, A.  \textit{Polyhedral approximations of Riemannian manifolds}, {Turkish J. Math.}, {27}, ({2003}), no.{1}, pp. {173--187},

\bibitem[P-2007]{petrunin-2}  Petrunin, A. \textit{Semiconcave functions in Alexandrov's geometry,} {Surveys in Differential Geometry}, \textbf{11} (2007) pp. 135--202.

\end{thebibliography}
\end{document}
