%title:  DC structure on Alexandrov space (AMS-tex file)
%author: g.perelman
%date: 4/14/95

\input amstex
\input amsppt.sty
\magnification=\magstep1
\hsize = 6.5 truein
\vsize = 9 truein
\NoBlackBoxes
\TagsAsMath
\NoRunningHeads

%The next three lines are so the logo doesn't print.
\catcode`\@=11
\redefine\logo@{}
\catcode`\@=13

%Section macros
\newskip\sectionskipamount
\sectionskipamount = 24pt plus 8pt minus 8pt
\def\sectionskip{\vskip\sectionskipamount}
\define\sectionbreak{%
	\par  \ifdim\lastskip<\sectionskipamount
	\removelastskip  \penalty-2000  \sectionskip  \fi}
\define\section#1{%
	\sectionbreak	%Encourages a page break, else inserts 24pt glue
	\subheading{#1}%
	\bigskip
	}
\font\tenbi=cmbxti10

\addto\tenpoint{\normalbaselineskip 18truept \normalbaselines}
\def\label#1{\par%
	\hangafter 1%
	\hangindent .5 in%
	\noindent%
	\hbox to .5 in{#1\hfill}%
	\ignorespaces%
	}

%Operator name macros
\define\op#1{\operatorname{\fam=0\tenrm{#1}}} %for text in math mode use
                                              %\op{...}

%QED box, from the TeXbook, p. 106.
\redefine\qed{{\unskip\nobreak\hfil\penalty50\hskip2em\vadjust{}\nobreak\hfil
    $\square$\parfillskip=0pt\finalhyphendemerits=0\par}}

	
	\define		\a		{\alpha}
	\redefine	\b		{\beta}
	\redefine	\d		{\delta}
	\redefine	\D		{\Delta}
	\define		\e		{\varepsilon}
	\define		\g		{\gamma}
	\define		\G		{\Gamma}
	\redefine	\l		{\lambda}
	\redefine	\L		{\Lambda}
	\define		\s		{\sigma}
	\redefine	\Sig  {\Sigma}
	\define		\p		{\partial}
\baselineskip=18 pt
\document

\centerline{\bf DC Structure on Alexandrov Space}
\centerline{(preliminary version)}
\centerline{\smc G.Perelman}

\bigskip

\vskip .5 truein

\specialhead{1. Generalities} \endspecialhead

{\bf 1.1 \ SC and DC functions on Euclidean spaces.} \ A function
$f: \Bbb R^n\to\Bbb R$ is called semiconcave (SC) if it is locally
representable as the difference
of a concave function and a smooth function.  Clearly SC is closed w.r.t.
addition,
multiplication by positive numbers and taking minimum. Notice also that if
$F:\Bbb R^\ell\to\Bbb R^m$ and $G:\Bbb R^m\to\Bbb  R^n$ have SC components
and the components of $G$ are increasing in each argument, then $G\circ F$
has SC
components.

We say that $f\in$ DC if it is locally representable as the difference of two SC
functions or, equivalently, as the difference of two concave functions. It
is easy
to see that DC is an algebra, and $f/g\in DC$ whenever $f,g\in$ DC and $g$
does not
vanish anywhere. Morever, if maps $F:\Bbb R^\ell\to\Bbb R^m$ and $G:\Bbb
R^m\to\Bbb
R^n$ have DC components then so does $G\circ F$. (Indeed, we can (locally)
decompose $F=F_2\circ F_1$, where $F_1:\Bbb R^\ell\to\Bbb R^{2m}$ has
concave components, and $F_2(x_1,\dots ,x_m,y_1,\dots ,y_m)=(x_1-y_1,\dots ,
x_m-y_m)$. Now $G\circ F_2$ has DC components, and we can write (locally)
$G\circ F_2=G_1-G_2$, where $G_1$ and $G_2$ have concave components,
increasing in each
argument. It follows that $G_1\circ F_1$ and $G_2\circ F_1$ have concave
components.)
A homeomorphism $F:\Bbb R^n\to\Bbb R^n$ can be called a DC isomorphism when
$f$ is
DC if and only if $f\circ F$ is DC. It follows from the previous remark
that $F$ is
a DC isomorphism iff $F$ and $F^{-1}$ have DC components.

\bigskip
{\bf 1.2 \ SC and DC functions on Alexandrov spaces.} \ Let $M^n$ be a (compact
when necessary) Alexandrov space with empty boundary.  A Lipschitz function
$f:M\to\Bbb R$ is SC if for each $x\in M$ there is a neighborhood
$U_x\ni x$ and $\lambda_x\in\Bbb R$ such that for every geodesic $\g$ in
$U_x$ the
function $f\circ\g(t)+\lambda_x t^2$ is concave in $t$. The basic example
of an SC
function is dist${}^2_x$ for some $x\in M$. Any continuous function $f$ can be
uniformly approximated by SC functions $f_j(x)=\inf_{y\in M} (f(y)+j\cdot
|xy|^2)$. If $f$ is SC in some domain $U\subset M$ and $K\subset U$ is
compact then
there is a function $\bar f$ which coincides with $f$ on $K$ and is SC on the whole
$M$. (Indeed, take $\bar f=\min(f,a{\op{dist}}^2_{\partial U}-b)$ on $U$
and $\bar f=a{\op{dist}}^2_{\partial U}-b$ on $M\backslash U$ for
appropriate large
$a,b$.)

We say that $f$ is DC on $M$ if it is locally representable as the
difference of two
SC functions. Our previous remark shows that the word "locally" can be dropped.
On the other hand, every DC function can be locally represented as the
difference of
two concave ones --- this follows from the existence of very concave
functions in
small neighborhoods of every point, see [P].

\bigskip
\specialhead{2. Background from [P] and an extension for general SC functions}
\endspecialhead

{\bf 2.1 \ Scalar product.} Every SC function has a differential at each point;
the differential $d_xf$ is a concave homogeneous function on the tangent
cone $C_x$,
and its restriction $f'_{(x)}$ to the space of directions $\Sig_x\subset C_x$ is
spherically concave. In particular, if $f=dist_y$, $y\neq x$, then
$f'{(x)}=-\cos{\op{dist}}_{y'}=: \chi_{y'}$, where $y'$ is the set of
directions of all
shortest geodesics from $x$ to $y$.

In general, if $\Sig$ is a compact Alexandrov space with curvature $\geq
1$, with
empty boundary, then a lipschitz function $f:\Sig\to\Bbb R$ is called
spherically
concave if $f(y)|xz|\geq f(x)|yz|+f(z)|xy|$ whenever $y$ lies on a shortest
geodesic
between $x$ and $z$. It is also convenient to consider 0-dimensional $\Sig$,
consisting of two points $x,y$ at the distance $\pi$, and say that
$f:\Sig\to\Bbb R$ is  spherically concave if $f(x)+f(y)\leq 0$.

Using induction on dimension, we can define a scalar product of two spherically
concave functions $f,g$ by
$$
\langle f,g\rangle =\sup_{x\in\Sig} (f(x)g(x)+
\langle f'_{(x)},g'_{(x)}\rangle) \ ,
$$
where the term with derivatives is dropped when dim $\Sig=0$. Obviously,
$\langle f,g\rangle=\langle g,f\rangle$, \
$\langle\lambda f,g\rangle=\lambda\langle f,g\rangle$ if
$\lambda\geq 0$, \ $\langle f,f\rangle\geq 0$ for any $f$. It is easy to check
by induction that
$$
\align
\langle\min (f,g),h\rangle & \leq \max(\langle f,h\rangle,\langle g,h\rangle) \\
\langle f+g,h\rangle & \leq \langle f,h \rangle +\langle g,h\rangle \\
\langle f,g\rangle^2 & \leq \langle f,f\rangle \langle g,g\rangle \\
\langle \chi_A,h\rangle & = -\min_{a\in A} h(a) \ \ {\text{ for any compact }}
   A\subset\Sig; \ \ {\text{in particular,}} \
\langle \chi_A,\chi_B\rangle =\cos|A,B| \\
\langle-f(a)\chi_a,g\rangle & = f(a)g(a)\leq \langle f,g\rangle \ \
  {\text{ if $f$ attains its minimal value at }} a \ .
\endalign
$$
Using quasigeodesics it is also easy to show by induction that
$\| f\| := \langle f,f\rangle^{\frac 12} =-\min_{x\in\Sig} f(x)$.

\bigskip
{\bf Remark}. In [P] we used a different scalar product which did not work for
general spherical concave functions.

\bigskip
{\bf 2.2 \ Consecutive approximations}. The scalar product, described in
2.1, can be
used to extend all the results of [P] by replacing the admissible functions and
functions of class DER with general SC and spherically concave functions
respectively. In particular, we have

\proclaim{Lemma} {\rm (cf. [P, Lemma 1])} Let $\Sig^{n-1}$ be a compact
Alexandrov space with curvature $\geq 1$, with empty boundary, and let
$f_i:\Sig\to\Bbb R$, $i=0,1,\dots ,k$ be spherically concave functions.
Assume that $\e=\min_{0\leq i\neq j\leq k} (-\langle f_i,f_j\rangle) > 0$.
Then
\roster
\item $k\leq n$, and
\item for each $i$, $1\leq i\leq k$, there exists $\xi_i\in\Sig$ such that
$f_j(\xi_i)=0$ for $j\neq 0,i$, \ $f_0(\xi_i)\geq \e$, $f_i(\xi_i)\leq-\e$.
\endroster
\endproclaim

Now let $f=(f_1,\dots f_k)$ be a map with SC components. A point $x\in M$
is called regular for $F$ if there exist $\e_x > 0$ and $U_x\ni x$ such that for
each $y\in U_x$ we have $\langle f'_{i(y)},f'_{j(y)}\rangle < -\e_p$
for all $1\leq i\neq j\leq k$, and there exists $\xi^+\in\Sig_y$ with
$f'_{i(y)}(\xi^+) > \e_p$ for all $1\leq i\leq k$. (In fact, the second
condition
needs to be checked only at $y=x$.) If $F$ is regular at $x$, then the
statement (2)
of the lemma allows us to use consecutive approximations to prove that $F$ is open
near $x$. (Indeed, we can increase all the coordinates of $F(y)$ by moving
in the
direction $\xi^+$, and we can decrease the $i$-th coordinate without
changing others
by moving in the direction $\xi_i$ guaranteed by the lemma.) In case $k=n$,
$F$ is
in fact a bilipschitz homeomorphism near $x$. (The proof of local one-to-one
property of $F$ is an easy argument based on volume and angle comparison and the
statement (1) of the lemma; in [P] it is hidden in the first step of
induction (from
$k=n+1$ to $k=n$) in the proof of the Main Theorem.)

\bigskip
\specialhead{3. DC coordinate charts}\endspecialhead

\proclaim{Proposition} Let $F=(f_1,\dots ,f_n)$ have SC components and be
regular in
some neighborhood of $x\in M$. Then
\roster
\item"(A)" If $f$ is SC near $x$ and $\langle f'_{(y)},f'_{i(y)}\rangle <
-\e_x < 0$,
  $1\leq i \leq n$, for all $y$ near $x$,\newline
  then $f\circ F^{-1}$ is SC near $F(x)$.
\item"(B)" If $\bar f$ is SC and increasing in each argument near $F(x)$ then
  $\bar f\circ F$ is SC near $x$.
\item"(C)" $f$ is DC near $x$ iff $f\circ F^{-1}$ is DC near $F(x)$.
\endroster
\endproclaim

{\smc Proof}. (A) Let $\bar\gamma(t)$ be a straight segment in the image of $F$,
$\gamma(t)=F^{-1}\circ\bar\gamma(t)$, \ $y=\gamma(0)$. From the fact that $F$ is
bilipschitz it is easy to see that $\gamma$ has unique right and left
tangent vectors
$\gamma^+(0),\gamma^-(0)\in C_y$. Furthermore, $d_y f(\gamma^+(0))+d_y
f(\gamma^-(0))\leq 0$. Indeed, otherwise, using concavity of $d_yf$ and
$d_yf_i$, we
could find a vector $v\in C_y$, such that $d_yf(v) > 0$ and $d_yf_i(v)\geq 0$,
$1\leq i\leq n$, which leads to a contradiction with the statement (1) of
the lemma
in 2.2. Now we'll check that

\roster
\item $f\circ\gamma (t)\leq f(y)+f'_{(y)} (\gamma^+(0))t+Ct^2$ when $t > 0$ is
small, for some $C$ independent of $t,y,\gamma$; the corresponding statement for
$t < 0$ and $\gamma^-(0)$ is checked similarly.
\endroster

Consider a quasigeodesic $\s$ starting at $y$ in the direction
$\gamma^+(0)/|\g^+(0)|$. Since $f,f_i$ are SC we have
\roster
\item"(2)" $f_i\circ\s(|\g^+(0)|t)\leq f_i(y)+f'_{i(y)} (\g^+(0))t+Ct^2$,
$1 \leq i\leq n$, and a similar inequality for $f$.
\endroster
(See [PPet]; in fact, in this argument we only need the first step of the
construction of quasigeodesics, which is not technically complicated.)
On the other hand,
\roster
\item"(3)" $f_i\circ\g(t)=f_i(y)+f'_{i(y)} (\g^+(0))t$, \ $1\leq i\leq n$ by the
definition of $\g$.
\endroster
Therefore, using the bilipschitz property of $F$, we can find a point $z$ in the
$Ct^2$-neighborhood of $\s(|\g^+(0)|t)$, such that
\roster
\item"(4)" $f_i\circ\g(t)\geq f_i(z)$, $1\leq i\leq n$, and
\item"(5)" $f(z)\leq f(y)+f'_{(y)}(\g^+(0))t+Ct^2$.
\endroster
We claim that
\roster
\item"(6)" $f(\g(t))\leq f(z)$
\endroster
Indeed, $z$ can be obtained from $\g(t)$ by consecutive approximations, as
in 2.2,
starting from $\g(t)$,and (4) guarantees that we only need to use the directions
$\xi_i$ in the process, thus increasing the value of $f$. Now (1) follows
from (5)
and (6).

(B) This is almost immediate from the definitions.

(C) This follows easily from (A) and (B). For example, if $f$ is SC near $x$
then\linebreak
$\tilde f=f+N {\op{dist}}^2_z$ satisfies
$\langle\tilde f'_{(y)},f'_{i(y)}\rangle < -\e_x < 0$ for all $1\leq i\leq n$
and all $y$ sufficiently close to $x$, if $z$ is obtained by moving $x$ a
little bit
in the direction where all $f_i$ increase, and $N$ is large enough. \qed

\bigskip
Let $S$ denote the set of  singular points of $M$, and let $M^*\supset
M\backslash
S$ be the set of all points $x\in M$ such that $\Sig_x$ contains $n+1$
directions
making obtuse angles with each other. $M^*$ is open, convex, and
${\op{dim}}_H(M\backslash M^*)\leq n-2$. (Convexity follows from Petrunin's work
[Pet] on parallel translation, the other properties follow from the results of
[BGP].) For each point $x\in M^*$ we can find a map $F:M\to\Bbb R^n$ with SC
components, which is regular near $x$. The collection of all such maps form an atlas
on $M^*$ and the transition functions are DC according to  statement (C) of our
Proposition. Following [OS] we can make the transition functions continuously
differentiable on $M\backslash S$ by taking the components of the
coordinate maps in
the form $\int_{y\in B}{\op{dist}}_y dH_n$.

\bigskip
\specialhead{4. Consequences}\endspecialhead

{\bf 4.0 \ Analytical preliminaries}. First we introduce some notation.
Let $F: U\subset M^*\to\Bbb R^n$ be a DC coordinate chart. We denote by
$DC_0$ the
class of DC functions on $F(U)$ which are continuously differentiable on
$F(M\backslash S)$, and by $BV_0$ the class of bounded functions of bounded
variation, which are continuous on $F(M\backslash S)$. At the end of the
previous
section we described a $DC_0$ atlas on $M^*$, and we can say that a
function $f$ is
$DC_0 \ (BV_0)$ near $x\in M^*$ if $f\circ F^{-1}$ is $DC_0 \ (BV_0)$ near
$F(x)$
for some (and hence for all) $DC_0$ charts $F$.

It is well known that the first partial derivatives of the $DC \ (DC_0)$
function
are in $BV \ (BV_0)$, and the second partial derivatives are signed Radon
measures,
with $\frac{\p^2 f}{\p x_i\p x_j} =\frac{\p^2 f}{\p x_j \p x_i}$ as measures.
We will also use a classical theorem of Alexandrov, which implies that DC
functions
have second differential a.e., and their first partial derivatives,
considered a.e.,
are differentiable a.e. It is also known (see [F, 4.5.9(29)]) that since
$H_{n-1}(S)=0$, we have $\|Df\|(A)=0$ for every $H_{n-1}-\s$-finite set $A$ and
$f\in BV_0$, in particular we can multiply any first partial derivative of
$f$ by a
bounded function which is continuous off a $H_{n-1}-\s$-finite set, and
still get
a signed Radon measure.

The following assertions will be used in 4.2 and 4.3. (see [V] for more general
results; my thanks to L.C.Evans for this reference)

\proclaim{Lemma}
\roster
\item If $f,g$ are bounded and BV then $fg$ is BV. Moreover, if $g$ does not
change sign and is bounded away from zero, then $f/g\in BV$.
\item If $f,g\in BV_0(U)$  then $(fg)_{x_i}=f_{x_i}\cdot g+
f\cdot g_{x_i}$ as measures.
\item Let $g\in BV_0(U)$, $U\subset\Bbb R^n_x$, and let
$F=(f_1,\dots ,f_n):U\to F(U)\subset\Bbb R^n_y$ be a $DC_0$ isomorphism. Then
$$
F_\#\left( \frac{\p g}{\p x_i} \ dx_1\wedge\dots\wedge dx_n\right) =
{\op{det}}(J_F)^{-1} \left( \sum_j
\frac{\p f_j}{\p x_i} \ \frac{\p(g\circ F^{-1})}{\p y_j}\right)
dy_1\wedge\dots\wedge dy_n \ .
$$
\endroster
\endproclaim

{\smc Proof}. (1) Clear by approximation, using semicontinuity of variation
measure.

(2) We need to check that $\int_U (fg)h_{x_i}=\int_U f_{x_i}\cdot gh+
\int_U f\cdot g_{x_i}h$ holds for any smooth function $h$ with compact support.
Of course this is true if both $f,g$ are smooth, and it is easy to check by
approximation if at least one of them is smooth. In general consider a
sequence of
mollified functions $g_j$; clearly $g_j$ converges to $g$ at each point
where $g$ is
continuous, which is $\|Df\|$--a.e. Thus $\int (fg_j)h_{x_i}\to\int
(fg)h_{x_i}$ and $\int f_{x_i}g_j h\to\int f_{x_i}gh$ by the dominated
convergence
theorem. To check the convergence of the remaining term, fix a small
$\delta > 0$
and let $K$ be a compact set where $f$ has jumps of size $\geq\delta$.
Since $\|Dg\|(K)=0$, we can find an open neighborhood $V\supset K$
such that
$\|Dg\|({\op{clos}} \ V) < \delta$ for large $j$. (Here we use that $\|Dg_j\|$
weaky converges to $\|Dg\|$.) Since $f$ has no jumps of size $>\delta$ near
$U\backslash
V$, we can find a continuous function $\bar f$ which is uniformly $2\delta$-close to
$f$ on
$U\backslash{\op{clos}} \  V$. Now
$$
|\int_U f\cdot g_{x_i}h-\int_{U\backslash{\op{clos}} \  V} \bar f\cdot
g_{x_i}h| \leq C\delta \ ,
$$
the same is true for $g_j$ with large $j$, and
$$
\int_{U\backslash{\op{clos}} \  V} \bar f(g_j)_{x_i}h\to
\int_{U\backslash{\op{clos}} \  V} \bar f g_{x_i}h
$$
because $\|Dg_j\|$ converges weakly to $\|Dg\|$.

(3) Arguing similarly to the proof of (2) we can cut off a neighborhood of
a compact
set where the first derivatives of the components of $F$ have jumps $\geq
\delta$,
then cover the rest of $U$ by small balls where those derivatives are nearly
constant, and check that the left- and right-hand sides of our identity are
nearly
equal on each ball,  using DC functions as test functions. The details are
left to
the reader.   \qed

\bigskip
{\bf 4.1 \ DeRham complex}. Differential forms on $M$ can be defined as
elements of
the vector space generated by monomials of the form $f_0 df_1\wedge\dots\wedge
df_m$, where all $f_i\in DC$, and two forms can be considered equivalent if they
have the same values a.e. The exterior differentiation can be defined by
$d(f_0df_1\wedge\dots\wedge df_m)= df_0\wedge\dots\wedge df_m$. Correctness
of this
definition follows from the identity
$$
df_0\wedge df_1\wedge\dots\wedge df_m(X_1\wedge\dots\wedge X_m)=
\sum^m_{j=0} (-1)^j \frac{\p}{\p x_j}(
f_0 df_1\wedge\dots\wedge df_m(X_0\wedge\dots\wedge \hat
X_j\wedge\dots\wedge X_m))
$$
for coordinate vectors $X_0,\dots ,X_m$ of some DC coordinate system $F$,
which holds at each point where all $f_i\circ F^{-1}$ are twice differentiable,
which is a.e.

\bigskip
{\bf 4.2 \  Metric tensor}. Let $F=(f_1,\dots ,f_m)$ be a $DC_0$ coordinate
chart near $x\in M^*$. Then, according to [OS], the metric of $M$ near $x$
can be
expressed by a metric tensor, defined and continuous at each nonsingular point.
Now let $f$ be a distance function, $f={\op{dist}}_y$, \ $y\neq x$. Then
$f\circ F^{-1}$ is DC  near $F(x)$, in particular, differentiable a.e., and
we have
$$
\sum_{i,j} g^{ij} \ \p(f\circ F^{-1})/ \p x_i\cdot
\p(f\circ F^{-1})/ \p x_j =1 \ \ {\text{a.e.}}   \tag 1
$$
Now suppose $x\in M^\delta$ for sufficiently small $\delta > 0$, where
$M^\delta=\{ x\in M: H_{n-1}(\Sig_x) > (1-\delta)H_{n-1}(S^{n-1})\}$ is an open,
convex subset of $M$, containing $M\backslash S$. Then we can choose a $DC_0$
coordinate chart $F$ and a collection of distance  functions $f^\a$, \
$1\leq \a\leq n(n+1)/2$, in such a way that the determinant of the system
of linear
equations (1) with $f^\a$ in place of $f$, with unknowns $g^{ij}$, is
positive and
bounded away from zero in a small neighborhood of $F(x)$. (Indeed, this is
easy to
arrange with some safety margin if $M=\Bbb R^n$, and the  condition $x\in
M^\delta$
guarantees that euclidean inequalities for
$\frac{\p (f^\a\circ F^{-1})}{\p x_i}$ continue to hold in $M$ up to a
small error.)
Thus the components of the metric tensor can be expressed as rational
functions of
the first derivatives of $f^\a\circ F^{-1}$. In particular, since
$f^\a\circ F^{-1}$ are DC near $F(x)$, we conclude that the components of
the metric
tensor are in $BV_0$ and differentiable a.e.

\bigskip
{\bf Remark}. This improves the earlier results of Otsu and Shioya [OS].

\bigskip
{\bf 4.3 \ Metric connection}. Let $A,B,C$ be bounded vector fields on
$M^\delta$,
and assume in addition that $A$ and $C$ are continuous off an
$H_{n-1}-\s$-finite
set, and $B \in BV_0$ (that is, the coordinates of $B$ in $DC_0$ charts are
$BV_0$).
Then there exists a signed Radon measure $\langle\nabla_A B,C\rangle$ on
$M^\delta$
which becomes
$$
\sum_{i,j,k} A^i C^j \left(
\frac{\p B^k}{\p x_i} \ g_{kj} +\tfrac 12 B^k
\left( \frac{\p g_{ij}}{\p x_k} +\frac{\p g_{kj}}{\p x_i} -
\frac{\p g_{ik}}{\p x_j}\right)\right) \cdot {\op{det}}(g_{ij})^{\frac 12}
dx_1\wedge\dots\wedge dx_n
$$
in each $DC_0$ coordinate chart $F: U\subset M^\delta \to\Bbb R^n_x$.
The correctness of the definition is proved by a standard computation using the
observations of 4.0.

\bigskip
{\bf 4.4 \ The Hessian of SC functions}.
\proclaim{Proposition} Let $F$ be a $DC_0$ chart near $x$ and let $f$ be a DC
function near $x$. Assume that $C_x=\Bbb R^n$, \ $f\circ F^{-1}$ has first and
second differentials at $F(x)$, and the components of the metric tensor
w.r.t. $F$
are differentiable at $F(x)$. Then $d_xf$ is linear on $C_x$ and there exists a
quadratic form $H_xf$ on $C_x$ such that
$$
f(y)=f(x)+d_xf(y')|xy| +\tfrac 12 H_xf(y',y')|xy|^2 +o(|xy|^2) \ ,
$$
where $y'\in \Sig_x$ denotes the direction of (any) shortest geodesic $xy$.
Moreover, $H_xf$ can be calculated using  standard formulas.
\endproclaim

{\smc Proof}. First of all, we can make a smooth change of coordinates in such a way
that in the new coordinate system $G$ the metric tensor at $G(x)$ becomes the
identity matrix, and its first derivatives vanish. (Indeed, consider a
smooth metric
with the same values of the metric tensor and its first derivatives at
$F(x)$, find
the coordinate transformation that produces normal coordinates, and apply
it to $F$.)
Thus we have $|g_{ij}(G(y))-\delta_{ij}|=o(|xy|)$, and therefore
\roster
\item $\left| \ |yz|-|G(y)G(z)| \ \right| =o(r^2)$ for $y,z\in B_x(r)$.
  Moreover, we have
\item $|\angle yxz-\angle G(y)G(x)G(z)| =o(|yz|)$ when all angles of the
triangle
  $G(y)G(x)G(z)$ are bounded away from zero.
\endroster
 Indeed, take a point $p$ such that $G(p)$ is in the plane $G(y)G(x)G(z)$, \
$G(x)$ is contained in the triangle $G(p)G(y)G(z)$ and all angles formed by
these
four points are bounded away from zero. Then (1) implies that
$$
\align
\tilde\angle yxz +\tilde\angle yxp +\tilde\angle zxp  & \geq
\angle G(y)G(x)G(z) +\angle G(y)G(x)G(p) +
\angle G(z)G(x)G(p) + o(|yz|) \\
&= 2\pi +o(|yz|) \ .
\endalign
$$
On the other hand, $\tilde\angle yxz\leq\angle yxz$, \
$\tilde\angle yxp\leq \angle yxp$, \
$\tilde\angle zxp\leq \angle zxp$, and
$\angle yxz+\angle yxp +\angle zxp\leq 2\pi$, whence
$|\tilde\angle yxz-\angle yxz|=o(|yz|)$ and (2) follows from (1).

Now let $y$ be close to $x$. We claim that the angle at $G(x)$ between the
directions of the straight segment $G(x)G(y)$ and the image of shortest geodesic
$G(xy)$ is $o(|xy|)$ --- clearly this proves the proposition.
To check the claim, find a point $y_1$ such that
$|xy_1|=|xy|/2$, the direction of $G(xy)$ is between that of $G(x)G(y)$ and of
$G(x)G(y_1)$,and the angle between the latter ones is $\pi/2$. Then (2)
implies that
$\angle(G(xy_1),G(x)G(y_1)) > \angle (G(xy),G(x)G(y))-o(|xy|)$. Thus if the
estimate
$\angle (G(xy),G(x)G(y))=o(|xy|)$ were false, we could construct a sequence
$y_i\to x$ with angles  $\angle (G(xy_i),G(x)G(y_i))$ bounded away from zero,
which is clearly impossible.  \qed

\bigskip
{\bf Remark}. In a recent work [O], Otsu proves a slightly weaker version
of this
proposition for distance functions. His technique is completely different,
and has
the advantage of expressing  Hessian of a distance function in terms of
derivatives
of the norms of Jacobi fields, which is a result of independent interest.

\bigskip\bigskip
\centerline{\bf References}
\bigskip
\label{[BGP]} Yu. Burago, M. Gromov and G. Perelman, Alexandrov spaces with
  curvature bounded below.  {\it Russ. Math. Surveys}, 1992.
\label{[F]} H. Federer, {\it Geometric Measure Theory}, 1969.
\label{[O]} Y. Otsu, Almost everywhere existence of second differentiable
structure
  of Alexandrov spaces, preprint.
\label{[OS]} Y. Otsu and T. Shioya, The Riemannian structure of Alexandrov
  spaces, {\it JDG}, 1994.
\label{[P]} G. Perelman, Elements of Morse theory on Alexandrov spaces,
  {\it SPb Math. J.}, 1994.
\label{[Pet]} A. Petrunin, Parallel transportation and second variation,
  preprint.
\label{[PPet]} G.Perelman, A.Petrunin, Quasigeodesics and gradient curves
on Alexandrov spaces, preprint. 
\label{[V]} A.I.Vol'pert, Spaces BV and quasi-linear equations, {\it Matem.
Sbornik} , 1967.
\enddocument
